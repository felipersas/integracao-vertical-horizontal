\documentclass[12pt,a4paper]{article}

% Pacotes de codificação e linguagem
\usepackage[utf8]{inputenc}
\usepackage[T1]{fontenc}
\usepackage[brazil]{babel}
\usepackage{csquotes}

% Pacotes de formatação
\usepackage[margin=2.5cm,top=3cm,bottom=2cm]{geometry}
\usepackage{setspace}
\usepackage{parskip}
\usepackage{titlesec}

% Pacotes gráficos e cores
\usepackage{graphicx}
\usepackage{xcolor}
\usepackage{booktabs}
\usepackage{longtable}

% Pacotes para links e referências
\usepackage[pdftex]{hyperref}
\usepackage{abntex2cite}

% Configuração de cores para links
\hypersetup{
    colorlinks=true,
    linkcolor=blue,
    filecolor=magenta,
    urlcolor=blue,
    citecolor=blue,
    pdftitle={Atividade 01: Análise da Integração Empresarial},
    pdfauthor={Aluno},
    pdfsubject={Integração Vertical e Horizontal},
}

% Configuração dos títulos das seções
\titleformat{\section}
  {\normalfont\Large\bfseries}{\thesection}{1em}{}
\titleformat{\subsection}
  {\normalfont\large\bfseries}{\thesubsection}{1em}{}
\titleformat{\subsubsection}
  {\normalfont\normalsize\bfseries}{\thesubsubsection}{1em}{}

% Espaçamento entre linhas
\onehalfspacing

% Início do documento
\begin{document}

% Capa
\begin{titlepage}
\centering
\vspace*{2cm}

{\Huge\bfseries Atividade 01\par}
\vspace{0.5cm}
{\LARGE\bfseries Análise da Integração Empresarial\par}
\vspace{1cm}
{\Large Estudos de Caso: Klabin S.A. e Marfrig S.A.\par}

\vspace{2cm}

\begin{flushleft}
\textbf{Disciplina:} Integração Vertical e Horizontal\\
\textbf{Professor:} Me. Deivison S. Takatu\\
\textbf{Data:} Fevereiro de 2026
\end{flushleft}

\vspace{2cm}

\begin{flushright}
\textbf{Aluno:} Felipe Marques\\
\textbf{Curso:} Análise e Desenvolvimento de Sistemas\\
\textbf{Instituição:} SENAI
\end{flushright}

\vfill

\end{titlepage}

% Sumário
\tableofcontents
\newpage

% Conteúdo da Atividade
\section{Introdução}

A presente atividade tem como objetivo analisar como empresas promoteem a integração da organização como um todo e os impactos dessas estratégias na cadeia de valor, coordenação de processos e tomada de decisão estratégica. Para tanto, foram selecionados dois casos relevantes do cenário industrial brasileiro: \textbf{Klabin S.A.} e \textbf{Marfrig S.A.}, empresas que adotam estratégias distintas de integração --- vertical e horizontal, respectivamente ---, ambas com impactos significativos em seus modelos de negócio.

A análise perpassará pela compreensão do posicionamento de cada empresa na cadeia de valor, a forma como coordenam seus processos e como as decisões estratégicas são tomadas a partir de suas estruturas integradas.

\section{Klabin S.A. --- Integração Vertical como Controle de Cadeia}

A Klabin exemplifica a \textbf{Integração Vertical} clássica no setor industrial brasileiro. A empresa controla múltiplos estágios da cadeia de valor, indo desde a floresta (matéria-prima) até a produção de papel e embalagens.

\subsection{Posicionamento na Cadeia de Valor}

Ao possuir e gerir seus próprios ativos florestais, a Klabin elimina a dependência de terceiros para o fornecimento de madeira, o insumo crítico da sua indústria. Isso cria uma \textbf{barreira de entrada} para concorrentes e estabiliza os custos de produção, um fator vital em um mercado de commodities.

A empresa atua como uma organização verticalmente integrada, controlando:

\begin{itemize}
    \item \textbf{Base Florestal:} Florestas plantadas de pinus e eucalipto para fornecimento de madeira.
    \item \textbf{Produção de Celulose:} Unidades industriais que transformam a madeira em celulose.
    \item \textbf{Fabricação de Papel:} Conversão da celulose em diversos tipos de papel.
    \item \textbf{Embalagens:} Produção de embalagens de papelão ondulado e sacos industriais.
\end{itemize}

\subsection{Coordenação dos Processos}

A integração vertical permite que o planejamento da colheita de madeira (extrativismo) esteja sincronizado com a demanda das máquinas de papel. O fluxo de informações é contínuo: se houver queda na demanda de embalagens, a empresa ajusta o manejo florestal, evitando o acúmulo de estoque de madeira.

Essa coordenação é facilitada por:

\begin{itemize}
    \item Sistemas de planejamento integrados que conectam o manejo florestal à produção industrial.
    \item Previsão de demanda de longo prazo que orienta o plantio e colheita.
    \item Logística interna otimizada entre florestas e unidades industriais.
\end{itemize}

\subsection{Tomada de Decisão Estratégica}

O controle total da cadeia permite que decisões de investimento em novas tecnologias sejam tomadas com base em garantias de suprimento de longo prazo, reduzindo riscos operacionais e financeiros.

Um exemplo emblemático é a unidade \textbf{``Puma''} em Ortigueira (PR), um dos maiores investimentos do setor de celulose e papel na história do Brasil. A decisão de investir aproximadamente R\$ 8,7 bilhões nesta unidade foi fundamentada na certeza do suprimento de matéria-prima proveniente das florestas próprias da empresa.

\section{Marfrig S.A. --- Integração Horizontal como Expansão de Mercado}

A Marfrig ilustra a \textbf{Integração Horizontal}, focada na aquisição e fusão com empresas que atuam no mesmo nível da cadeia de valor.

\subsection{Posicionamento na Cadeia de Valor}

Ao adquirir concorrentes (como a operação da BRF em mercados específicos ou empresas internacionais como a Keystone Foods e Moy Park), a Marfrig expandiu drasticamente sua participação no mercado e escala de produção. Isso aumentou seu poder de barganha frente a grandes varejistas globais (como McDonald's).

A estratégia horizontal da Marfrig é caracterizada por:

\begin{itemize}
    \item \textbf{Aquisições Internacionais:} Keystone Foods (EUA) e Moy Park (Reino Unido).
    \item \textbf{Expansão de Portfólio:} Incorporação de marcas e produtos de empresas adquiridas.
    \item \textbf{Escala Global:} Posicionamento como um dos maiores processadores de proteína animal do mundo.
\end{itemize}

\subsection{Coordenação dos Processos}

A integração horizontal permite a \textbf{padronização de processos} entre diferentes unidades. As melhores práticas de uma planta em uma região podem ser replicadas globalmente. Além disso, a logística de distribuição é otimizada, pois a empresa passa a controlar múltiplos pontos de entrada em mercados chave (América do Sul, Europa, Ásia).

A coordenação é viabilizada por:

\begin{itemize}
    \item Protocolos operacionais padronizados em todas as unidades.
    \item Sistemas de gestão integrados que permitem visibilidade global.
    \item Centros de excelência que disseminam melhores práticas.
\end{itemize}

\subsection{Tomada de Decisão Estratégica}

A estratégia horizontal reduz a competição direta e permite acessar portfólios de produtos mais amplos. As decisões estratégicas focam em \textbf{sinergias}: redução de custos administrativos, unificação de marcas em mercados onde antes eram concorrentes e aumento da capacidade de exportação.

A aquisição da Keystone Foods, por exemplo, garantiu à Marfrig o fornecimento exclusivo para o McDonald's em diversos mercados, consolidando sua posição como fornecedor global de proteína animal para a rede de fast food.

\section{Comparativo e Análise Integrada}

Ao comparar as estratégias de Klabin e Marfrig, observa-se que ambas buscam aumentar a eficiência e a competitividade, por meio de caminhos distintos:

\begin{table}[h]
\centering
\begin{tabular}{@{}p{5cm}|p{5cm}|p{5cm}@{}}
\toprule
\textbf{Aspecto} & \textbf{Klabin (Vertical)} & \textbf{Marfrig (Horizontal)} \\
\midrule
\textbf{Foco Principal} & Controle da cadeia produtiva & Expansão de mercado e escala \\
\midrule
\textbf{Benefício Central} & Estabilização de custos e garantia de fornecimento & Poder de barganha e diversificação \\
\midrule
\textbf{Risco Associado} & Alto investimento em ativos fixos & Desafios de integração cultural \\
\midrule
\textbf{Vantagem Competitiva} & Barreiras de entrada e previsibilidade & Acesso a novos mercados e clientes \\
\bottomrule
\end{tabular}
\caption{Comparativo entre as estratégias de integração}
\end{table}

\section{Conclusão}

Enquanto a Klabin usa a integração vertical para garantir \textbf{eficiência e controle do insumo}, a Marfrig usa a integração horizontal para garantir \textbf{domínio de mercado e escala}. Ambas, no entanto, utilizam essas estratégias para criar uma visão sistêmica do negócio, onde decisões operacionais impactam diretamente os resultados financeiros globais.

A escolha entre integração vertical ou horizontal depende das características específicas de cada setor, dos recursos disponíveis e da estratégia competitiva almejada. No caso apresentado, ambas as empresas demonstram sucesso em suas respectivas abordagens, comprovando que não há um modelo único de integração --- mas sim estratégias que devem ser alinhadas ao contexto e aos objetivos organizacionais.

\section{Referências Bibliográficas}

\begin{enumerate}
    \item \textbf{KLABIN.} Relatório Anual e de Sustentabilidade 2023. Disponível em: \url{https://www.klabin.com.br/ri/}. Acesso em: fev. 2026.

    \item \textbf{MARFRIG.} Relatório Anual e de Sustentabilidade 2023. Disponível em: \url{https://marfrig.com.br/ri/}. Acesso em: fev. 2026.

    \item \textbf{TAKATU, Deivison S.} Aula 02 --- Integração Vertical e Horizontal nas Empresas. Apresentação, 2024.

    \item \textbf{PORTER, Michael E.} Vantagem Competitiva: Criando e Sustentando um Desempenho Superior. Rio de Janeiro: Campus, 1989.

    \item \textbf{CHANDLER, Alfred D.} Strategy and Structure: Chapters in the History of the Industrial Enterprise. Cambridge: MIT Press, 1962.
\end{enumerate}

\end{document}
