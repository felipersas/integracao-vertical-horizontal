\documentclass[12pt,a4paper]{article}

% Pacotes de codificação e linguagem
\usepackage[utf8]{inputenc}
\usepackage[T1]{fontenc}
\usepackage[brazil]{babel}
\usepackage{csquotes}

% Pacotes de formatação
\usepackage[margin=2.5cm,top=3cm,bottom=2cm]{geometry}
\usepackage{setspace}
\usepackage{parskip}
\usepackage{titlesec}

% Pacotes gráficos e cores
\usepackage{graphicx}
\usepackage{xcolor}
\usepackage{booktabs}
\usepackage{longtable}

% Pacotes para links e referências
\usepackage[pdftex]{hyperref}
\usepackage{abntex2cite}

% Configuração de cores para links
\hypersetup{
    colorlinks=true,
    linkcolor=blue,
    filecolor=magenta,
    urlcolor=blue,
    citecolor=blue,
    pdftitle={Relatório Técnico: Transformação Digital na Siderurgia},
    pdfauthor={Aluno},
    pdfsubject={Integração Vertical e Horizontal},
}

% Configuração dos títulos das seções
\titleformat{\section}
  {\normalfont\Large\bfseries}{\thesection}{1em}{}
\titleformat{\subsection}
  {\normalfont\large\bfseries}{\thesubsection}{1em}{}
\titleformat{\subsubsection}
  {\normalfont\normalsize\bfseries}{\thesubsubsection}{1em}{}

% Espaçamento entre linhas
\onehalfspacing

% Início do documento
\begin{document}

% Capa
\begin{titlepage}
\centering
\vspace*{2cm}

{\Huge\bfseries Relatório Técnico\par}
\vspace{0.5cm}
{\LARGE\bfseries Transformação Digital na Siderurgia\par}
\vspace{1cm}
{\Large Análise da Integração Vertical e Horizontal na Gerdau S.A.\par}

\vspace{2cm}

\begin{flushleft}
\textbf{Disciplina:} Integração Vertical e Horizontal\\
\textbf{Professor:} Me. Deivison S. Takatu\\
\textbf{Data:} Fevereiro de 2026
\end{flushleft}

\vspace{2cm}

\begin{flushright}
\textbf{Aluno:} [Seu Nome Completo]\\
\textbf{Curso:} [Seu Curso]\\
\textbf{Instituição:} [Sua Instituição de Ensino]
\end{flushright}

\vfill

\end{titlepage}

% Sumário
\tableofcontents
\newpage

% Conteúdo do Relatório
\section{Introdução}

A indústria contemporânea está passando por uma revolução silenciosa, porém profunda, impulsionada pela necessidade de competitividade em um cenário globalizado e digitalizado. Conforme abordado na disciplina ``Integração Vertical e Horizontal'', a manufactura moderna deixou de ser composta por máquinas isoladas operando em ``ilhas de automação'' para se tornar um ecossistema conectado, onde a informação é o ativo estratégico mais valioso.

Neste contexto, a integração de sistemas deixa de ser uma opção tecnológica para se tornar um requisito estratégico. A \textbf{integração vertical}, que articula os diferentes níveis operacionais do chão de fábrica até a gestão estratégica, e a \textbf{integração horizontal}, que conecta toda a cadeia de valor desde fornecedores até o cliente final, são fundamentais para garantir a agilidade e a precisão na tomada de decisão.

O presente relatório apresenta um estudo de caso real sobre a \textbf{Gerdau S.A.}, maior produtora de aços longos das Américas. O objetivo é analisar como a implementação desses conceitos de integração favoreceu a empresa, impulsionando sua eficiência operacional, reduzindo custos e assegurando sua liderança em um mercado de commodities de alta competitividade. A análise perpassará pelos fundamentos tecnológicos, sistemas corporativos e o uso estratégico de dados industriais.

\section{Perfil da Empresa}

A Gerdau é uma empresa brasileira com atuação global, líder no mercado de aços longos nas Américas e uma das principais fornecedoras de aços especiais do mundo. Fundada em 1901, no Rio Grande do Sul, a empresa consolidou seu crescimento através de uma combinação de gestão sólida e capacidade de inovação.

\subsection{Dados Reais e Contexto Atual}

\begin{itemize}
    \item \textbf{Faturamento (2023):} R\$ 78,5 bilhões, demonstrando robustez mesmo em cenários econômicos desafiadores.
    \item \textbf{Capacidade Produtiva:} Aproximadamente 26 milhões de toneladas de aço por ano.
    \item \textbf{Presença Geográfica:} Operações industriais em 10 países e atuação comercial em mais de 20 nações.
    \item \textbf{Colaboradores:} Cerca de 30 mil profissionais ao redor do mundo.
    \item \textbf{Modelo de Negócio:} A Gerdau atua desde a produção de aço até a transformação e distribuição de produtos siderúrgicos, atendendo setores como a construção civil, indústria automotiva, agroindústria e energia.
\end{itemize}

Um ponto central da estratégia recente da Gerdau é o programa \textbf{Gerdau Next}, que visa redefinir o futuro da companhia através da transformação digital, sustentabilidade e inovação. É dentro deste programa que as integrações vertical e horizontal ganham destaque.

\section{Integração Vertical na Gerdau: Do Forno à Estratégia}

A Integração Vertical na Gerdau consiste na articulação efetiva entre o nível de controle (chão de fábrica) e o nível corporativo (gestão). Na siderurgia, esse desafio é monumental devido à complexidade dos processos de fabricação de aço.

\subsection{Automação e Fundamentos Tecnológicos}

As usinas da Gerdau, como a Usina Riograndense (RS) e Ouro Branco (MG), operam com altos índices de automação. O processo siderúrgico envolve altíssimas temperaturas e reações químicas complexas. Para garantir a qualidade do aço, a empresa utiliza:

\begin{itemize}
    \item \textbf{Sistemas de Controle Distribuído (DCS):} Para o controle global do processo de fabricação.
    \item \textbf{CLPs (Controladores Lógicos Programáveis):} Para o controle local de máquinas específicas, como laminadores e fornos.
\end{itemize}

A integração vertical ocorre quando os dados gerados por esses dispositivos (nível 0 e 1 da pirâmide da automação) não ficam presos nos painéis de operação, mas são transportados via redes industriais até os sistemas de supervisão (SCADA) e, posteriormente, aos sistemas de gestão empresarial (ERP -- \textit{Enterprise Resource Planning}).

\subsection{Manutenção Preditiva e IIoT}

Um exemplo prático dessa integração é o uso da \textbf{Internet das Coisas Industriais (IIoT)}. A Gerdau instalou sensores inteligentes em ativos críticos, como motores de grandes dimensões, redutores e transformadores.

Esses sensores coletam dados de vibração, temperatura e corrente elétrica em tempo real. Antigamente, essa informação seria monitorada apenas localmente. Hoje, através da integração vertical, os dados são enviados para uma nuvem corporativa onde algoritmos analisam a ``saúde'' do equipamento.

\subsubsection{Benefício}

Isso permite a \textbf{Manutenção Preditiva}. Em vez de consertar a máquina quando ela quebra (reativo) ou fazer manutenção em intervalos fixos (preventivo), a empresa intervém exatamente quando necessário, baseada em dados. Isso reduz paradas não planejadas e aumenta a vida útil dos ativos, impactando diretamente o resultado financeiro (nível estratégico).

\section{Integração Horizontal na Gerdau: Cadeia de Valor Conectada}

Enquanto a integração vertical foca na eficiência interna, a integração horizontal na Gerdau foca na fluidez dos processos ao longo da cadeia de valor: fornecedores, produção, logística e clientes. O objetivo é eliminar ``silos organizacionais'' onde departamentos como compras, vendas e logística agiam de forma independente.

\subsection{Integração com Fornecedores e Reciclagem}

A Gerdau é o maior reciclador da América Latina. O principal insumo de suas usinas elétricas (Fornos a Arco Elétrico) é a sucata ferrosa (ferro velho). A integração horizontal permite conectar os fornecedores de sucata diretamente ao sistema de planejamento da produção da usina.

Isso garante que o fluxo de matéria-prima seja sincronizado com a necessidade de fundição, evitando filas de caminhões na portaria e garantindo que os fornos nunca parem por falta de material.

\subsection{Logística e Distribuição Integrada}

Na ponta da venda, a integração horizontal é fundamental para a experiência do cliente. Através da conexão entre o sistema de vendas (CRM) e o sistema logístico (WMS -- \textit{Warehouse Management System} e TMS -- \textit{Transportation Management System}), a Gerdau consegue oferecer transparência total.

Quando um cliente faz um pedido de vergalhões para uma obra, o sistema verifica automaticamente o estoque no centro de distribuição mais próximo, aloca a carga e define a rota de entrega otimizada.

\subsection{Gerdau Marketplace}

Um marco na integração horizontal foi o lançamento do \textbf{Gerdau Marketplace}, uma plataforma digital B2B. Essa ferramenta integra a base de clientes da construção civil diretamente à frota logística e ao estoque da empresa. O cliente pode comprar, agendar entrega e rastrear o pedido online, criando uma ponte direta que elimina intermediários burocráticos e acelera o fluxo de informações.

\section{Dados Industriais e Inteligência de Negócios (BI)}

Conforme visto na disciplina, a ``Informação como Ativo Estratégico'' é crucial. Na Gerdau, a massa de dados gerada pela integração vertical e horizontal é tratada através de ferramentas de Business Intelligence (BI) e Inteligência Artificial.

\subsection{Otimização do Mix de Produtos}

Através da análise de dados históricos de venda e custos de produção, a empresa utiliza algoritmos para sugerir o melhor mix de produtos a ser fabricado em cada usina. Por exemplo, analisar se é mais eficiente produzir aço para a construção civil em uma usina específica baseando-se no consumo de energia e disponibilidade de insumos naquele momento.

\subsection{Rastreabilidade e Qualidade}

Para clientes de segmentos exigentes, como a indústria automotiva, a rastreabilidade é mandatória. A integração de sistemas permite que a Gerdau rastreie um vergalhão ou uma peça de aço até a corrida de fundição específica. Se houver uma não-conformidade, a empresa consegue identificar exatamente quais lotes foram afetados, agilizando recalls e garantindo a segurança.

\subsection{Eficiência Energética}

A energia elétrica representa um dos maiores custos na produção de aço. Através do monitoramento integrado dos dados dos medidores de energia em todo o parque fabril, a empresa consegue identificar padrões de consumo e automatizar o funcionamento de equipamentos para operar em horários de menor tarifa ou menor demanda, resultando em economia de milhões de reais anuais.

\section{Conclusão}

O estudo de caso da Gerdau S.A. ilustra claramente a teoria apresentada na disciplina: a indústria moderna deve operar como um ecossistema conectado.

A aplicação da \textbf{Integração Vertical} permitiu que a Gerdau transformasse o chão de fábrica, tradicionalmente isolado, em uma fonte rica de dados para a tomada de decisão estratégica, possibilitando a manutenção preditiva e o controle preciso da produção. Já a \textbf{Integração Horizontal} foi responsável por alinhar a operação com o mercado, conectando a coleta de sucata à entrega do produto final, eliminando barreiras entre departamentos e criando valor para o cliente através de plataformas digitais.

O resultado dessa abordagem integrada é refletido na solidez financeira da empresa e na sua capacidade de inovar. A Gerdau deixa de ser vista apenas como uma siderúrgica tradicional e se posiciona como uma empresa de tecnologia industrial, provando que a transformação digital é o caminho inevitável para a competitividade na Indústria 4.0.

\section{Referências Bibliográficas}

% Usando citações manuais em formato bibliográfico

\begin{enumerate}
    \item \textbf{GERDAU.} Relatório Anual e de Sustentabilidade 2023. Disponível em: \url{https://www.gerdau.com.br/investidores/relatorio-anual}. Acesso em: fev. 2026.

    \item \textbf{GERDAU.} Programa Gerdau Next - Transformação Digital. Disponível em: \url{https://www.gerdau.com.br/pt-br/inovacao}. Acesso em: fev. 2026.

    \item \textbf{TAKATU, Deivison S.} Apresentação da Disciplina e Contextualização: Integração Vertical e Horizontal. Aula 01, 2024.

    \item \textbf{LARA, Carla Eduarda Orlando de Moraes de.} Automação e controle industrial. Curitiba: Contentus, 2021.

    \item \textbf{CAIÇARA JÚNIOR, Cícero.} Sistemas integrados de gestão: ERP – uma abordagem gerencial. Curitiba: Intersaberes, 2015.
\end{enumerate}

\end{document}
